\chapter{WORKING DATA}
This project bases all mechanisms on one single birdsong representation, the
source field recordings.
This section describes the data used, how it is collected, and prepared for
usage within the program.

\section{Working Data Format}
Birdsong is 

\section{Xeno-Canto Database}
Xeno-Canto (ref) provides a substantial number of field recordings of various
bird species, both calls and songs.
Recordings are available with varying levels of quality, ranging from extremely
clear with minimal perceivable noise, to extremely noisy recordings.
Recordings may have single or multiple instances of the same species, and/or
different species.

note on copyright

\subsection{Automatic Sample Retrieval}
Manual selection and download of recordings is expensive in terms of time.
An automatic method is desireable, although there exists no public facing API\.
We developed a web scraper specifically for automatic retrieval.

The scraper allows the specification of:
\begin{itemize}
  \item Species to filter;
  \item Recording quality to filter;
  \item Retrieval interval for continuous operation.
\end{itemize}

\subsection{The Use of Metadata}
Xeno-canto provides the following metadata with each recording:
\begin{itemize}
  \item Date and time
  \item Recording location
  \item Species recorded
  \item Existence of other species
\end{itemize}

This program makes use only of the prominent species tagged in the recording.
The location of the recording could be used to improve the accuracy of the
program, as certain species are restricted to certain parts of the world.
This greatly affects the probability of a certain species being identified in
a recording, as long as location metadata is present.

\section{Preparation}
Recordings are resampled to 22000 khz to reduce the memory footprint and
processing power required to operate on each recording.
Resampling to 22000 khz was found to have no significant reduction in quality
or information retained, despite the high frequency vocalizations in birdsong.

All recordings are processed into spectrograms through a fast fourier transform
(FFTS) method provided by the xxx python library.
This representation provides a visualization of energy present in each frequency
band in function of time.
Each frequency is quantized into discrete bands according to the parameters set.
Time is quantized into etc.
The absolute energy is preserved, we don't lose any information, we gain it.
\textbf{show spectrogram image under waveform of same section}

The following parameters are used:
\begin{itemize}
  \item hamming window: ur mom
\end{itemize}

See Appendix for details on FFTS.

Frequencies above x and below y are removed from the spectrograms as these do
not contain signals belonging to any bird species \textbf{(cite)}.
