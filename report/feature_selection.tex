\section{Identifying Useful Features}

Orthonologists use this and that.
We're taking an image recognition based approach

show spectrograms of various species to show differences

Variations in amplitude along the song are not taken into account but may be a 
useful feature to consider.

Direct spectral information such as mean energy per frequency bin is not taken
although this can be a useful statistic to help identify the species.

\section{Template Selection}
\subsection{Basic Elimination}
Some effort is taken to reduce the number of templates extracted in the
preprocessing stage to reduce the computational and storage overhead otherwise
incurred.
An increase in template count implies an increase in noise and inconsistencies
in the semantics captured. \textbf{is semantics the right word?}

\textbf{is RF good at ignoring noise? I think so.?}

\textbf{speculative section on what makes a good template}

\textbf{for extraction section: several extra templates are included around the template}

The following criteria is used to select valid templates:
\begin{itemize}
\item dimensions within x
\item uhh
\end{itemize}
These criteria was reached by empirical trial and error.

\subsection{Guided Elimination}
It is desireable to reduce the template count further by recognizing aspects which
make for good templates. This subsection outlines some speculative options for
selecting better templates, and reducing those which would have a low importance
score after training and classification.

\subsubsection{Image contrast}
It can be argued that templates with low local contrast contain insufficient
information to be meaningful in any way during template matching.
Templates with a low contrast match against much of any image, resulting in an
increase in noise.
Implicitly \textbf{is implicitly the right word?} such templates have a high
correlation with not only the species from which it was extracted but with all
species.
\textbf{show some images, maybe graphs of correlation}

\subsubsection{Spatial inclusion}
Due to the imperfect nature of the preprocessing methods used, gaps and
inconsistencies in structure appear in the thresholded spectrogram.
These inconsistencies are present also in repeated components in a bird song,
at all levels of granularity.
This causes multiple templates to be extracted for a single component in some
instances, and single larger blocks to be extracted in others.

In many of these cases, one template's bounding box intersects or is contained
entirely within another template's bounding box.
Merging these templates by extracting the bounding box of the union of the two
or more templates may result in more consistent extractions.
\textbf{show some images}

Similarly, templates which are sufficiently close to eachother may be merged,
but care must be taken not to form extremely large templates.

\subsubsection{Variation in granularity}
There exists a variance in granularity for the extracted templates, in which
some sections of song are mostly connected to form a single template, and others
are disconnected, leaving templates with single syllables \textbf{right word?}
and templates with entire sections of song.
This is a similar to the observation in \textbf{spatial inclusion}.

\textbf{what to do about it}
\textbf{is it a problem}


\subsubsection{inter-template correlation}
some templates may correlate with eachother, should they be merged?

It can also be argued that templates with little to know intercorrelation may be
independent anomalies such as noise in the signal or other sounds irrelevant to
the subject species.

\subsubsection{Species-specific template statistics}
It may be possible to use information regarding average dimensionality and mean
frequency information to determine the relevance likelyhood of a particular template.
Such metrics require an existing set of validated templates, which may be gathered
by filtering a non discriminated set of templates by their measures importances.
