\section{Feature Engineering}
With the collection of extracted templates from each spectrogram,
a model can now be constructed for a particular sample.
This model is represented as a feature vector consisting of the maxima
of each template cross correlation operation done on the spectrogram.
This vector will then be compared to those of other samples using a
classifier.
This section describes in detail the operation of template matching
to build this feature vector, as well as some notes on the time
required.

\subsection{Cross-correlation Mapping}
Cross-correlation mapping, also referred to as template matching, is
a method for determining the similarity of an image within another,
often larger image.
It is essentially a form of image recognition.
The intuition is that songs from birds of the same species will have
extremely similar spectral shapes.

Cross-correlation mapping works by convolving the template image
over the target image and measuring the pixel similarities.
check this.

The Open-CV library is used to perform template matching.
Open-CV's implementation is highly optimized, and may be computed
using a GPU. For details see appendix.

\subsubsection{Results}
The result of template matching is a cross-correlation mapping of the
template against the target spectrogram.
show an example of a template ccm against a spectrogram.

The data in the mapping can give us a rough estimate of how closely
the template matches.
Given this result, we store the global maxima of the mapping in a
feature vector.

\subsection{Feature Vector}
When classifying a particular sample, the spectrogram is cross-correlated
with each template accumulated so far in the database.
For each cross-correlation, the maxima of the result is taken and stored
at the index in the vector correspoinding to the template that was used.
We then refer to this index later for further analysis.

\subsection{Computational Expense and Optimizations}
Template matching is the most expensive operation in the program.
Although the underlining algorithm is itself well optimized, further
improvements can be made for marginal gains.

\subsubsection{Time anal}
Template matching takes approximately x minutes per template, given
mean dimensions of mxn and ixj for spectrograms and templates respectively.
Considering the quantity of templates stored in the database, the time
required quickly compounds into the order of days.
xx templates against yy spectrograms was measured at x days on a xyz machine.
This stresses the requirement for optimization, which is the topic of
section blah.

\subsubsection{Implemented and proposed optimizations}
Dimensionality reduction:
Correlation area truncation:
