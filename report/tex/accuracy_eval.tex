\section{Accuracy Evaluation}

a mention on differences in measuring accuracy for single-label vs multi
label classifier

mention any metrics that we do not use but are standard, and explain
why we do not use them.

\subsection{Validation Strategy}
In effort to reduce sensitivity to chance selection of random samples, all
measurements are performed using stratified 10-fold cross validation.
This gives us a 90/10 train/test split, and is repeated with a new random sample
shuffle 10 times.

Randomness can be seeded for repeatability.

\subsection{Accuracy Metrics}
mean accuracy, f-score, etc
results

\subsection{AUC curve}
we dont use this but if we can construct one then show how we do and what
this tells us

\subsection{Confusion Matrix}
A confusion matrix plots the true-positive rate against the false-positive rate
of each class.
It illustrates the rate of confusion between a given class and all others.

In this case it is instrumental to show the correlation between bird song of
different species, and shows indications of their similarity of their
vocalisations on the level of granularity defined by the templates extracted.

Figure xyz shows the confusion matrix for the samples picked.

... matrix here ...

The confusion matrix shows good classification performance for most species.\\

species A has the highest error-rate, being confused mostly for species B.
This shows that their templates, and therefore their song, share some
similarities.\\

species C has an excellent score, having been confused with no other species.
This shows that species C has very unique vocal characteristics amongst the
picked samples.
