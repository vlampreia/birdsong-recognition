\section{Our Approach}
The approach taken for this project is inspired by similar sound recognition
tasks.
Such tasks often use the spectrographic representation of an audio sample and
use image recognition to identify key elements in the structure.

Our approach uses a combination of computer vision and machine learning
techniques to construct a fully automatic recognition system.
Standard image processing methods are used to process spectrograms and extract
sections of song which may be used to identify a particular species, much like
how an orthonologist visually inspects the song spectra.
These sections are then matched against new samples to be classified through a
multi-class machine learning algorithm.

move bulk of subsections to appendices

\subsection{Process Overview}
The project is divided into four descrete parts.
These follow the logical flow of data:
\begin{enumerate}
  \item \textbf{Collection:}
    Data is sourced from field recordings done in uncontrolled environments.
    The variety provides a good estimate of real-world performance and
    introduces many quality related issues.

  \item \textbf{Preparation and selection:}
    Recordings are filtered and selected to maintain reasonable quality levels.
    Spectrograms are then derived from the recordings.
    This is now the representation that is used throughout the program until
    the feature vector is constructed.

  \item \textbf{Preprocessing and feature extraction:}
    Noise is reduced as much as possible to identify key regions of interest
    within the spectrogram image.
    These are extracted as templates and cross-correlated against other
    recording spectrograms to form a feature vector.

  \item \textbf{Classification and evaluation:}
    The resulting data is then fed to a classifier and evaluated using techniques
    designed to reduce statistical bias.
\end{enumerate}

Each of these procedures are described in detail in their respective sections.
A few of these sections discuss possible alternatives or improvements to the
developed mechanisms.

A regular focus of the project is performance statistics, which is done for each
stage of the program to some granularity.
A report on this is available in appendix xyz.

\subsection{Architecture Overview}
\textbf{diagram of key sections}

Due to the experimental nature of the project, a rigorous architecture has not
been designed ahead of time.
The codebase evolved through several iterations 

\subsubsection{Data persistence}


\subsection{Implementation Technologies}
All code is written in Python 2.7.
Libraries used include:
\begin{itemize}
  \item open-CV 2.0
\end{itemize}
