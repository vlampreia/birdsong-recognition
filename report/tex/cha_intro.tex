\chapter{INTRODUCTION}

\section{Motivation}
Automatic birdsong recognition is far from solved outside of controlled
environments or with heavy user input.
A system which can recognize species from birdsong from a wide variety of
species is an exciting application of signal processing, computer vision and
machine learning technologies.
Such software would be beneficial to the bird enthusiasts and environmental
health associations, and others.

\section{Goals}
The primary aim of this project is to research and develop potential methods
for automatic birdsong recognition, with little to no user interaction.
The program should recognize up to 50 distinct bird species from field recordings
provided by arbitrary sources around the world.
The recordings should not have any form of rigorous quality control such as
recordings performed under controlled environments.
Sample selection may be performed to ensure quality levels are kept at a
reasonably normalised level, as many recordings are expected to contain
excessive amounts of noise.

For the purposes of this project, bird songs will be solely considered, and calls
will not be used.
This limits the complexity of the classification requirements as calls tend to
differ significantly to songs.
