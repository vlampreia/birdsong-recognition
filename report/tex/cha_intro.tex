\chapter{INTRODUCTION}

\section{Motivation}
Current manual and semi-automatic methods of bird species recognition are limited
by the cost and challenge of analysing enormous amounts of field recordings.
Identifying a species from thousands of other possibilities is a challenge itself
considering not only the vast variety of birds but also the similarities
present between certain species.
For these reasons large-scale verification tasks are limited in scope, and results
are often not immediate.
A fully automatic mechanism for identifying bird species from song recordings is
therefore highly valuable for ornithology.

Examples of the immediate usage of such observation technologies include
population monitoring and migration tracking of species in the fields of
biogeography and conservation efforts.

The application of automatic birdsong recognition is also of interest to the
amateur bird watcher or the generally curious.

The problem itself touches on many facets of computer science and engineering,
including digital signal processing and analysis, image recognition,
machine learning, and performance optimization.

\section{Goals}
The primary aim of this project is to research and develop potential methods
for automatic birdsong recognition, requiring little to no user interaction.
The program should be able to identify the species of the most prominent bird
performing in a song recording.

The recordings used originate from arbitrary locations and sources around the
world.
They shall not undergo any form of manual selection or processing.
In similar vein, they should also not be subjected to rigorous quality control
aside from what classifications may be provided by sources.
This is to ensure that validation represents real-world performance.

For the purposes of this project, some limitations are imposed to simplify some
problem areas.
Firstly, A general level of quality is ensured by selecting only higher quality recordings
as defined by the data source.
This is a reasonable restriction which reduces the number of recordings required.
Secondly, the number of species the system should know about is limited to 50
randomly picked labels.
Thirdly, only bird songs are considered.

\section{Bird Songs vs Calls}
Bird songs differ from calls in complexity, length and context.

song is usually but not always performed by males.

this shouldnt be a section
