\chapter{INTRODUCTION}

\section{Motivation}
Current manual and semi-automatic methods of bird species recognition are limited
by the cost and challenge of analysing enormous amounts of field recordings.
Identifying a species from thousands of other possibilities is a challenge itself
considering not only the vast variety of birds but also the similarities
present between certain species.
For these reasons large-scale verification tasks are limited in scope, and results
are often not immediate.
A fully automatic mechanism for identifying bird species from song recordings is
therefore highly valuable for ornithology.

Prominent examples of the immediate usage of such observation technologies include
population monitoring and migration tracking of species in the fields of
biogeography and conservation efforts.

The application of automatic birdsong recognition is also of interest to the
amateur bird watcher.

The problem itself is of great interest and may involve multiple approaches
including digital signal processing, analysis, image recognition, machine learning,
...

\section{Goals}
The primary aim of this project is to research and develop potential methods
for automatic birdsong recognition, with little to no user interaction.
The program should recognize up to 50 distinct bird species from field recordings
provided by arbitrary sources around the world.
The recordings should not have any form of rigorous quality control such as
recordings performed under controlled environments.
Sample selection may be performed to ensure quality levels are kept at a
reasonably normalised level, as many recordings are expected to contain
excessive amounts of noise.

For the purposes of this project, only bird songs will be considered, and calls
will not be used.
This limits the complexity of the classification requirements.
No preseletion of species is to be done for the general validation tests, except
for some additional test cases where species with very similar calls will be
validated.
