\section{Classification}
With feature vectors constructed, we select a machine learning algorithm to
train and classify birdsong based on our model.
This section details the choice of algorithm, it's variations, performance and
tuning.

\subsection{Approaches to Classification}
obviously a classification task.
We are working with limited sample set, but high number of features.
The data is labelled, so we will select a supervised learning algorithm.
we must select a classifier which performs well with this data.
identifying the best classifier can be best done by cross validating each and
selecting the best performer.
For the purposes of testing our approach, a single algorithm was initially
chosen.

random forest is a good performer and is highly scalable (ref).
We chose this algorithm.

also might look at gradient boosted trees.

number of classifiers can be used
skim through some strengths and weaknesses

\subsection{Random Forest Classifier}
A random forest classifier is an ensamble of trees.
A node of a tree is split on an index of the feature vector.
Each Tree uses a new random sampling of the features.

we pick the random forest.
explain why we use this classifier
explain how it fits with the data we are working with

This is a multi-class classifier, returning the probabilities of each class
corresponding to the data given.
An alternative approach is possible, in which the classification of each species
is split into individual random forests, the final result of which is retrieved
by taking the maximum of all probabilities.
Using multiple binary classifiers allows other accuracy metrics to be used,
facilitating the accuracy evaluation of single species.

\subsection{Extremely Randomised Trees}
Extremely Randomised Trees is similar to a traditional Random Forest, however
however splitting is randomised, instead of being computed for optimal
performance.
This has the benefit of being faster, with the drawback of being more sensitive
to noisy features.\\

Using this classifier has seen no major accuracy or performance differences.
why do we use it, no reason??

ert thresholds are random, best one is picked, reduces variance, increases bias

\subsection{Parameter Selection}
show parameters available for RF
show initial parameters chosen and their performance
how did we pick these? (make something up)

Section~\ref{sec:tuning} explores semi-automatic parameter tuning and touches on
the issues of overfitting.
