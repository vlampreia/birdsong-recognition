\chapter{CONCLUSIONS}

%- a summary and critical evaluation of what has been achieved
%
%- some feedback on what you personally have learned from the project
%
%- some thoughts to what you would do with more time (what are the
%  most important tasks not carried out?), and also how you would
%  change your approach if you were tarting over, based on what you
%  know now.

\section{Summary}
This report describes the solution developed for the task of automatic birdsong
recognition.
Existing approaches to relevant problems were used as a starting point.
The approach reduces the initial problem to image recognition using spectrograms
of the 2211 field recordings sourced from Xeno-canto.
Image processing techniques were investigated to maximise template quality.
A random forest classifier was selected to classify 240 of the samples using the
cross-correlation results.
After tuning, the model was evaluated using 10-times 10-fold cross validation, 
achieving a peak accuracy of 88\%, and F1 score of 87\%.
Finally, feature importances were analysed to rank templates and perform further
feature reductions to improve scalability.

\section{Reflection}
An open-ended project such as this has scope for exploratory research.
Regardless of the approach taken, the nature of the project touches on many
facets of computer science.
Our approach focuses on computer vision and machine learning, touching on
performance computing with (relatively) large amounts of data.\\

Although good results were achieved, these could have been improved.
Model performance is highly dependent on the features used, and this area
lacks in sophistication and validation.
Although we learned about a few computer vision techniques for this, we could
have increased the breadth and depth of our investigations in this area.

We are confident that the choice of classifier was a good one.
A lot was learned about machine learning methodologies, although there exists
a lot of contradictory findings and advice, it is the general consensus that
with whatever application, it is the data that shapes the approach and the only
conclusive evaluation to be made is directly based on this.

Given more time, other classifiers could be evaluated as well,
however validation was limited immensely by the lack of data processed data.
A total of 20 samples per label was not enough to confidently measure the
performance of the classifier.
Given more data, a further separation of development and validation could have
been done, leading to a more representative evaluation.\\

The lack of data stems from the immense processing power required to
cross-correlate all 35773 templates.
This makes it extremely expensive to evaluate new template extractions.
With this in mind it would have been justifiable to seek external computing
resources early on, and to spend more time on optimisations.
With this in mind, Python was not the right choice in terms of performance.\\


On the topic of software engineering, recommended practices were not observed.
The exploratory nature of the project has lead to many quick iterations and hacks
in the software.
Some effort was expended to restructure large portions of the code to make it
more robust to future changes, however this quickly deteriorated.
This was due to the fact that no strict requirements were produced for the
program's functionality.
It was always inteded as a test-bed for the explored methods, and so there exists
a tradeoff between developing the framework and developing the solution.
We have learned instead that simplicity is key: equal performance would have
been achieved by splitting the program into discrete artefacts, and transferring
data through files instead of in-memory.


\section{Future Work}


%You currently do not at all address standard SE issues in terms of the
%design, implementation and testing of the software you produced. This
%is fine with me because I am much more interested in what you have
%written about. I still felt I should point this out in case you want
%to squeeze in a summary.
%For this I plan to include some diagrams to describe the flow of data and
%how the software is structured.
%I am yet to write about my development process, but it will not be
%anything too in depth.
%I also intend to discuss any difficulties or matters of interest
%retrospectively in the conclusion.
%Some SE matters are briefly discussed in some of the texts, but I did not
%want to dive into too much detail since implementation details are not the
%main concern.
%No strict testing took place. Something to reflect on again.
