\section{Preparation}\label{sec:prep}
\subsection{Resampling}
Recordings are resampled to 22000 khz to reduce the memory footprint and
processing power required to operate on each recording.
Resampling to 22050 khz was found to have no significant reduction in quality
or information retained, despite the high frequency vocalizations in birdsong.

The right channel of any stereo recordings is discarded.

The source MP3 files are converted into 22kHz 16-bit WAV files.

Resampling and conversion is done using ffmpeg and automated through a shell
script.


\subsection{Spectrogram Construction}
All recordings are processed into spectrograms through a fast fourier transform
(FFTS) method provided by the xxx python library.
This representation provides a visualization of energy present in each frequency
band in function of time.
Each frequency is quantized into discrete bands according to the parameters set.
Time is quantized into etc.
The absolute energy is preserved, we don't lose any information, we gain it.
\textbf{show spectrogram image under waveform of same section}

The following parameters are used:
\begin{itemize}
  \item NFFT: 512
  \item window: Hamming window of 512
  \item n overlap: 75\%
\end{itemize}

See Appendix for details on FFTS.

Frequencies above 10000Hz and below 100Hz are removed from the spectrograms as these do
not contain signals belonging to any bird species \textbf{(cite)}.
