\section{Data Source: Xeno-Canto}

Xeno-Canto (ref) provides a substantial number of field recordings of various
bird species, both calls and songs.
Recordings are available with varying levels of quality, ranging from extremely
clear with minimal perceivable noise, to extremely noisy recordings.
They are of variable length, with an average duration of 5 minutes.
Individual or multiple instances of the same species, and/or
different species may be present in a single recording.
Recordings are available as MP3 files of varying quality.

note on copyright


\subsection{Automatic Sample Retrieval}
Manually selecting and downloading recordings is a time-consuming process.
A public API does not exist for Xeno-canto, therefore we developed a web scraper
specifically for automatic retrieval.

The scraper allows the user to filter on species and recording quality before
downloading a sample.
Species filtering may be done by exlusion or exlusive inclusion.

When repeated fetching is required, an interval may be set in order to reduce
strain on Xeno-canto's servers.
Once the interval has been set, the scraper will continuously download samples
at the specified rate until it has been interrupted by the user.\\

The webscraper is written in Python 2.7 using the XXXX package.

\subsection{Metadata}
Xeno-canto provides the following metadata with each recording:
\begin{itemize}[noitemsep]
  \item Date and time
  \item Recording location
  \item Species recorded
  \item Existence of other species
\end{itemize}

This program only makes use of the prominent species tagged in the recording.
Although not used in our implementation, the location of the recording could be
used to improve the accuracy of the classifier, as certain species are restricted
to certain parts of the world.
This greatly affects the probability of a certain species being identified in
a recording, as long as location metadata is present.
Special care has to be taken however to consider migration patterns of specific
species.
