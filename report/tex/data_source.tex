\section{Data Source: Xeno-Canto}

Xeno-Canto (ref) provides a substantial number of user uploaded field recordings
of various bird species, both calls and songs.
Recordings may originate from any part of the world, may vary extremely in
terms of quality, and any number of species may be present in a single
recording.

Recordings are of varying duration, ranging from 1 to 20 minutes each.
The content may be densely packed, with multiple birds singing simultaneously,
or sparse with long durations of silence.

Xeno-canto provides all audio as dynamic MP3.
These are normalised as described in Section~\ref{sec:prep}

\subsection{Metadata}
Xeno-canto provides the following metadata with each recording:
\begin{itemize}[noitemsep]
  \item Date and time
  \item Recording location
  \item Species recorded
  \item Existence of other species
\end{itemize}

This program only makes use of the prominent species tagged in the recording.
Although not used in our implementation, the location of the recording could be
used to improve the accuracy of the classifier, as some species are restricted
to certain parts of the world.
Including this information in the feature set is very likely to provide a
good accuracy by itself, but without much precision.
It can be used however to narrow down the set of possible species, so long as
special care is taken to account for migration patterns.


\subsection{Automatic Sample Retrieval}
Manually selecting and downloading recordings is a time-consuming process.
A public API does not exist for Xeno-canto, therefore we developed a web scraper
specifically for automatic retrieval.

The scraper allows the user to filter samples on species and on recording quality
before downloading a sample by examining the metadata present in the HMTL.
Filtering may be done by exclusion or selection.

When repeated fetching is required, an interval may be set in order to reduce
strain on Xeno-canto's servers.
Once the interval has been set, the scraper will continuously download samples
at the specified rate until it has been interrupted by the user.\\

The webscraper is written in Python 2.7 using the |lxml| package.
