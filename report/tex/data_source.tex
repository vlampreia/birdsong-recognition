\section{Data Source: Xeno-Canto}

Xeno-Canto \parencite{xenocanto} provides over 353830 recordings of 9728
individual species.
Recordings consist of both calls and song, uploaded both professional and
amateur users, and
may originate from any part of the world, vary extremely in
terms of quality, and also content.

Recordings vary in duration, ranging from a few seconds to over 10 minutes.
The content may be densely packed, with multiple birds vocalising simultaneously,
or sparse with long durations of silence.
Additionally, multiple species may be present in a single recording, however
higher quality recordings tend to contain a singly identifiable species, and
are otherwise labeled.

Xeno-canto provides all audio as dynamic MP3.
These are normalised as described in Section~\ref{sec:prep}

\subsection{Metadata}
Xeno-canto provides the following metadata with each recording:
\begin{itemize}[noitemsep]
  \item Date and time
  \item Recording location
  \item Species recorded
  \item Existence of other species
\end{itemize}

This program only makes use of the prominent species tagged in the recording.
Although not used in our implementation, the location of the recording could be
used to improve the accuracy of the classifier, as some species are restricted
to certain parts of the world.
Including this information in the feature set is likely to provide a
some accuracy by itself, but perhaps without much precision.
It could however be used to narrow down the set of possible species, so long as
special care is taken to account for migratory patterns.
This has not been done.

\subsection{Automatic Sample Retrieval}
Manually selecting and downloading recordings is a time-consuming process.
A public API does not exist for Xeno-canto, therefore we developed a web scraper
specifically for automatic retrieval in |Python 2.7| using the |lxml| package.

The scraper allows the user to filter samples on species and on recording quality
before downloading a sample by examining the metadata present in the HMTL.
Filtering may be done by exclusion or selection.

When continuous fetching is desired, an interval may be set in order to reduce
strain on Xeno-Canto's servers.
Once an interval is set, the scraper will continuously download samples
at the specified rate until it has been interrupted by the user.\\

The scraper was ran throughout the course of development, for an approximate
accumulated time of 40 hours, with an average interval of 60 seconds between
downloads.
2211 audio samples were collected, totalling just over 45 hours.
See Appendix~\ref{app:files} for a comprehensive list of all recordings which
have been used including relevant attributions.
