\section{Useful Feature Identification}

Orthonologists use this and that.
It is observed that birdsong varies not only between species but also between
birds of the same species.
We're taking an image recognition based approach

It has been shown that birds learn their song as they age (ref), which may pose
an issue to our classification algorithm if we don't have enough samples.
On the other hand, it is plausable that given enough samples, we can estimate
the age group of a bird in a given recording by how much it's song correlates
to other known samples, without the need for labels indicating age.
It is not known if this is an issue that affects this project or not, as our
data source does not provide the age of the bird as metadata.

show spectrograms of various species to show differences

Variations in amplitude along the song are not taken into account but may be a 
useful feature to consider.

Direct spectral information such as mean energy per frequency bin is not taken
although this can be a useful statistic to help identify the species.

A potentially useful feature is average syllable (continuous volcalisation)
duration, however it can be argued that this information is implicitly included
by template matching.
