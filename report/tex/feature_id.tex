\section{Useful Feature Identification}
This section explores the information given by the spectrogram image
representation of an audio file, and what features may be extracted to
characterise and identify bird species.

\subsection{Song Segment Formalisms}
Bird song structure is well defined in terms of segmentation \parencite{Catch1997}.
Segment labelling is defined in a naturally descending order of granularity, from
large sequences to basic singular elements, and are identified by their length
and duration of silence between sounds:
\begin{itemize}
  \item \textbf{song sequence:}
    An entire song is a conplete end-to-end sequence of multiple or a single
    phrase.
    The same phrase is may be repeated exactly or with variation within a
    song.
  \item \textbf{phrase:}
    A phrase consists of a series of usually equal syllables.
    Phrases at the end of a song are often not composed of equal syllables.
  \item \textbf{syllable:}
    Syllables may be simple or complex vocalisations.
    Complex syllables can be further partitioned into individual elements.
  \item \textbf{element:}
    Elements form the most basic of vocalisations, for instance, sweeps or tones.
\end{itemize}

%\begin{figure}[!htb]
%  \centering
%  \begin{subfigure}[t]{0.5\textwidth}
%    \centering
%    \caption{}
%  \end{subfigure}
%  \begin{subfigure}[t]{0.5\textwidth}
%    \centering
%    \caption{}
%  \end{subfigure}
%  \caption{Spectrograms visualising formalised song segments}
%\end{figure}

\subsection{Distinctive Features in Spectrograms}
It is common knowledge that bird song is genereally consistent amongst species,
and differ from species to species.
However, there exist some inconsistencies which may become problematic if our
system does not have enough samples to account for the variance.
Specifically, variations exist within a species songs between individuals.

One instance of such variation is regional, where the geographical
location of birds has been found to correlate with nuanced differences in detail.
Local isolation of populations also plays a part in minute variations of song,
referred to as dialects \parencite{podos2007}.

It has also been observed that birds learn aspects of their song, which affects
the exactness of the reproduction \parencite{Krood1983}.
Recordings of younger birds is therefore expected to differ, possibly to a
significant extent.
This is an interesting subject in itself: it is plausable that an automatic
system may be constructed which identifies not only the species of bird but also
an estimate of its age by similar cross-correlation methods.

Birds are able to produce a large varienty of sounds.
These vary from simple pure tones to complex harmonics with amplitude and
frequency modulations \parencite{fager2004}.

Our image recognition approach makes direct use of the structures present
in bird songs.
The intuition is that spectrograms contain all the information necessary to
distinguish a bird song from another, and that these may be segmented and
catalogued such that they form the basis of truth for a statistical
classification system.
Such a system is essentially the automatic equivalent of manual spectrogram 
analysis performed by skilled ornithologists, and somewhat an analogue of natural
pattern recognition done when observing sounds in the field.

Section~\ref{sec:granularity} discusses the issue of selecting an appropriate
granularity for use in cross-correlation.

\subsection{Potential Alternative and Additional Features}
There are a number of potentially useful features which have not been tested or
measured directly in this project.
These are not limited to image recognition approaches, however some are
implicitly included in our approach.

\subsubsection{Variations in amplitude}
Amplitude variations are present in many bird vocalisations.
These have been seen to be dynamic and dependant on the environment and social
context \parencite{brumm2004}
It is therefore likely that this feature is not significantly important for
species recognition.

This feature is however included in the cross-correlation approach since
amplitude is represented in spectrograms, although bearinbg inconsistencies due to
both the dynamic nature of the amplitude variations, and the granularity of the
extracted segments.

\subsubsection{Statistical analysis}
It is possible to extract information regarding the energy distribution in the
spectrogram directly.
Such information may be useful as the difference in song results in observable
differences in fundamental frequencies and bandwidth.
Additionally, it is possible that certain harmonics are characteristic to a
limited number of unique species.

Although not likely to be useful for direct classification of species due to the
existence of similarities between some species, these features may be helpful in
narrowing down the set of possible species, which would save significant time by
reducing the number of cross-correlations needed to exhaustively search for
matching species.

\subsubsection{Segment lengths and repetition frequencies}
Segment statistics such as minimum, maximum and mean duration may be useful to
help identify bird species.
The silence between segments, as well as the repetition rate may also be a
unique or indicative characteristic.

These features are implicitly included through image recognition, but only at
the level of granularity afforded by the template extraction mechanism.
The direct inclusion of extensive segment statistics has been shown to boost
the classification accuracy to some significance \parencite{lasseck2013}.
