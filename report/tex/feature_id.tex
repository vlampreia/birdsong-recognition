\section{Useful Feature Identification}
This section discusses what information shown in a spectrogram can be used as
useful features. ...

\subsection{Segment Identification}
The structure of bird song is composed of discrete elements which have specific
taxonomy.
The following labelling and partitioning is common:
song sequence, phrase, syllable, element \textbf{show sgram}

\subsubsection{Song sequences}
are ... groups of phrases or single phrase

\subsubsection{Phrases}
are...

\subsubsection{Syllables}
...

\subsubsection{Elements}
....


\subsection{Consistencies Between Birds}
It is observed that birdsong varies not only between species but also between
birds of the same species.

It has been shown that birds learn their song as they age (ref), which may pose
an issue to our classification algorithm if we don't have enough samples.
On the other hand, it is plausable that given enough samples, we can estimate
the age group of a bird in a given recording by how much it's song correlates
to other known samples, without the need for labels indicating age.
It is not known if this is an issue that affects this project or not, as our
data source does not provide the age of the bird as metadata.

show spectrograms of various species to show differences


\subsection{Fragment Extraction}
Our image recognition approach makes direct use of the vocalisation structures
present in bird songs.
The intuition is that sampling song segments at a specific granularity captures
the acoustic profile of a specific species, which can then be used to compare
against spectrograms of other instances of bird song.

say something about the content of these segments...? but what

Section~\ref{sec:granularity} touches on the topic of granularity selection.



\subsection{Potentially Useful Features}
There are a number of other potentially useful features which have not been
tested or measured directly in this project.
These are not limited to an image recognition approach.

\subsubsection{Variations in amplitude}
Some species have characteristic amplitude variations throughout their song,
or within phrases or syllables.
This is captured implicitly through image recognition on these segments.
The length of the variation is of course dependent on the level of granularity
captured by the extraction method: if syllables are extracted, amplitude
variations along phrases are not captured.

\subsubsection{Statistical analysis}
It is possible to extract information regarding the energy distribution in the
spectrogram directly.
For example, it is known that different species have different frequency bands
in which they vocalize \textbf{ref}.
It is therefore possible to measure the vocal bandwidth, mean or dominant
frequency bands, to help distinguish birds by narrowing down the set of
possible species.

These measurements are not likely to isolate a single species by themselves,
however they may be helpful in reducing the number of possible species and thus
reducing the overall computational cost by reducing the number of image
recognition tasks.

\subsubsection{Segment lengths and repetition rates}
A potentially useful feature is average syllable (continuous volcalisation)
duration, however it can be argued that this information is implicitly included
by template matching.
