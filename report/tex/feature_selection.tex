\section{Template Extraction}
Before we can extract template images for cross-correlation, they must first be
identified in the source spectrograms.
For this to be viable at any level of consistency, spectrograms must undergo
noise reduction and segmentation.
This section details the mechanisms developed for preprocessing spectrogram
images and filtering segments so that only relevant templates are included for
extraction.

Each spectrogram is processed in sequence using functionality provided by the
Python OpenCV library, and immediately stored on disk.
Parallelisation is possible here, but the computational cost of this stage is
not large enough to justify sparing development time on this aspect at the time.

\subsection{Spectrogram Preprocessing}\label{sec:preproc}

Before any features can be extracted, their pixel boundaries must be found.
This is a non-trivial problem due to noise and recording artefacts
that may be present in each spectrogram.
The complexity of the problem is risen by the inconsistency of these from sample
to sample.

Figure~\ref{fig:sgram_noise} shows an example of a noisy spectrogram.
Notice how background noise makes it difficult to find the exact boundary of
some of the vocalisations.
\textbf{define noise, both background and unwanted signal also rewrite the notice}

\begin{figure}[h]
  \centering

  \caption{Considerably noisy spectrogram of a segment of xbird song.}
  \label{fig:sgram_noise}
\end{figure}

Even after quality prefiltering done when downloading recordings from
Xeno-canto, noise levels remain at undesireable levels.

Removing noise and unwanted artefacts manually is precise, although tedious.
Because a high number of samples is required for our approach, we developed an
automatic noise reduction stage, as doing so manually would be far too
impractical at this scale.
Standard computer vision techinques for noise reduction were used, as well as
techniques for discrete object identification.

\subsubsection{The mechanism}
A filter is first applied to the spectrogram, which reduces the noise in the
image by smoothing just enough until granular noise is reduced sufficiently
(Figure~\ref{fig:preproc_vis_filter}).
This is accomplished using a Gaussian filter with a 5x5 kernel and sigma of 0.
\textbf{sigma 0??}
Median filtering was also tested. \textbf{so what happened?}

The image is then thresholded using a thresholding algorithm
(Figure~\ref{fig:preproc_vis_thresh}).
Otsu's binarization thresholding technique was used to select an optimal
threshold value.
Because of pixel intensity inconsistencies throughout individual spectrograms,
specifically in areas where noise is prominent, an adaptive thresholding
algorithm was also tested.
This did not appear to provide generally better results.

Further noise reduction is then performed using dilation and erosion
(Figure~\ref{fig:preproc_vis_dilation1}),
which removes small segments and joins pixel groups which are in close proximity
to each other.
Notice that the order of operations is flipped here since we are working with
an inverted spectrogram.
Dilation and erosion is then performed a second time for larger segments
(Figure~\ref{fig:preproc_vis_dilation2}),
A kernel size of 3x3 and 7x7 were used for small and large segments, respectively.

Most remaining holes are then filled using a closing morphology algorithm,
using an ellipse of size 3x3 (Figure~\ref{fig:preproc_vis_closing}).

\begin{figure}[h]
  \centering
  \begin{subfigure}[b]{0.55\textwidth}

    \caption{Gaussian filter, $k=5x5, \sigma=0$}
    \label{fig:preproc_vis_filter}
  \end{subfigure}
  \begin{subfigure}[b]{0.55\textwidth}

    \caption{Otsu's threshold}
    \label{fig:preproc_vis_thresh}
  \end{subfigure}
  \begin{subfigure}[b]{0.55\textwidth}

    \caption{Dilation followed by erosion, $k=3x3$}
    \label{fig:preproc_vis_dilation1}
  \end{subfigure}
  \begin{subfigure}[b]{0.55\textwidth}

    \caption{Dilation followed by erosion, $k=7x7$}
    \label{fig:preproc_vis_dilation2}
  \end{subfigure}
  \begin{subfigure}[b]{0.55\textwidth}

    \caption{Closing, 3x3 ellipse}
    \label{fig:preproc_vis_closing}
  \end{subfigure}
  \caption{Preprocessing stages in order.}
  \label{fig:preproc_vis}
\end{figure}

Following these procedures the spectrogram is transformed into a binary image
consisting of discrete pixel segments.
This representation simplifies the extraction procedure as described in
Section~\ref{sec:extract}.


\subsubsection{Granularity considerations}\label{sec:granularity}
It is important to consider and evaluate the granularity to aim for when
isolating sections of song.
It is not clear if large sections of song would perform better than smaller,
individual vocalisations.
An evaluation should therefore be performed to determine the best granularity.

Trialing different granularities bears the weight of template matching for all
new templates, which is extremely time consuming.

Further, achieving consistent granularity across all spectrograms is not a trivial
task, and is certainly not possible if a single parameter set is used for
all spectrograms.

A loose aim is therefore taken to extract the smallest possible non-singular
vocalisations.
This of course does not always work, but the developed mechanism gives good
results.

\subsubsection{Quality and consistency}
It is difficult to arrive at a single set of optimal parameters that work
well across all spectrograms.
An adaptive method is therefore suggested (but not implemented):
Preprocessing parameters may be specified on a per-recording basis if prior
knowledge is gathered for expected vocalisation/section counts, template
dimensions, and frequency range.
Since there is no known structured source of data for this, it is necessary to
manually specify these on a per-species basis.
Alternatively a fully automatic method is conceivable by feeding back the
classification results with each granularity setting.
This however would be incredibly time consuming on standard hardware, and may
be sensitive to other factors in the classification pipeline.

Consistent quality becomes less of a concern as the number of samples used for
training increases.
It can be shown that as the sample count increases, the number of valid templates
tends to increase.
The number of noise also increases, however these do not intercorrelate as valid
templates do, and are handled well by less sensitive classifiers such as
random forests (\textcite{marko2004}).

Quality is therefore moderated after preprocessing in the selection stage as
described in Section~\ref{sec:template_select}.


\subsection{Basic Selection and Extraction}\label{sec:template_select}
With optimal results the segments represent parts of song at the desired
granularity, from elements to phrases.
This is unfortunately not the case.
Noise may persist after preprocessing, and new deformations may appear, such as
incorrectly joined or incomplete segments.

Some effort is taken to reduce the number of undesireable templates.
In addition to noise, undesireable templates include extremely specific or
extremely generalised templates.

For example, if a sound is featured in a single instance of bird song, but not
in any other recording of songs of that species, then it is an extreme outlier,
which may occur due to a bird's individual expression or external sources.
It is non trivial to detect outliers during feature extraction, but because
random forests are generally insensitive to this type of outlier, we do not
treat these in this stage.

On the other hand, generalised templates are easier to detect.
Such templates will have a very faint structure (low contrast), or a complete
lack of structure.
These may be analysed directly and ignored, as they are likely to correlate,
although very weakly, to the majority of spectrograms irrespective of the
species.

Although these rejections bring a theoretical boost to classifier accuracy and
performance, they mainly benefit performance by reducing the number of
cross-correlations computed.

\subsubsection{Selection mechanism}
Complete elimination of all undesireable features is impossible without also
removing desireable features.
A balance must therefore be struck, in both preprocessing and selection stages.
Performing selection manually is extremely time consuming given the quantity
of pixel sections.
Selection is therefore done automatically by considering the dimensionality and
pixel values of each template.\\


Templates with the following properties are automatically rejected:
\begin{itemize}[noitemsep]
  \item \textbf{Area smaller than 50 pixels:} Templates with small dimensions
    are too small to be of any significance and are most likely noise or
    artefacts from preprocessing;

  \item \textbf{Area larger than 10000 pixels:} Templates with extremely large
    dimensions are most likely artefacts from preprocessing, high intensity
    noise or continuous external noises.
    It is possible for sections of song to be merged together into a single
    pixel region due to preprocessing errors.

  \item \textbf{Minimum pixel intensity greater than the average for the
    spectrogram:} attempt at reducing "boring" templates
    but it doesnt make sense
\end{itemize}


\subsubsection{Extraction mechanism}\label{sec:extract}

A bounding box is computed for the contours of each segment, which is then used
to extract a template from the original spectrogram image.
Once selection is performed for a particular pixel region, it's bounding box is
expanded by 10 pixels in each direction, and that region is extracted from the
original, unprocessed spectrogram.
This forms a single template.
The template is then blurred using Gaussian filtering with a sigma of 1.5.

The reason for blurring is to reduce the dimensionality of the template, which
brings performance boosts to template matching (described in
Section~\ref{sec:ccm}) since the template may now be reduced in size.

Blurring the template also improves it's generality, which makes matches more
likely.
Special consideration must be taken in order to ensure templates are not overly
generalised, as this will cause excess ambiguity amongst templates.



\subsection{Guided Template Elimination}
It is desireable to reduce the template count further by recognizing aspects which
make for good templates. This subsection outlines some speculative options for
selecting better templates, and reducing those which would have a low importance
score after training and classification.

\subsubsection{Image contrast}
It can be argued that templates with low local contrast contain insufficient
information to be meaningful in any way during template matching.
Templates with a low contrast match against much of any image, resulting in an
increase in noise.
Implicitly \textbf{is implicitly the right word?} such templates have a high
correlation with not only the species from which it was extracted but with all
species.
\textbf{show some images, maybe graphs of correlation}

\subsubsection{Spatial inclusion}
Due to the imperfect nature of the preprocessing methods used, gaps and
inconsistencies in structure appear in the thresholded spectrogram.
These inconsistencies are present also in repeated components in a bird song,
at all levels of granularity.
This causes multiple templates to be extracted for a single component in some
instances, and single larger blocks to be extracted in others.

In many of these cases, one template's bounding box intersects or is contained
entirely within another template's bounding box.
Merging these templates by extracting the bounding box of the union of the two
or more templates may result in more consistent extractions.
\textbf{show some images}

Similarly, templates which are sufficiently close to eachother may be merged,
but care must be taken not to form extremely large templates.

\subsubsection{Variation in granularity}
There exists a variance in granularity for the extracted templates, in which
some sections of song are mostly connected to form a single template, and others
are disconnected, leaving templates with single syllables \textbf{right word?}
and templates with entire sections of song.
This is a similar to the observation in \textbf{spatial inclusion}.

\textbf{what to do about it}
\textbf{is it a problem}


\subsubsection{inter-template correlation}
some templates may correlate with eachother, should they be merged?

It can also be argued that templates with little to know intercorrelation may be
independent anomalies such as noise in the signal or other sounds irrelevant to
the subject species.

\subsubsection{Species-specific template statistics}
It may be possible to use information regarding average dimensionality and mean
frequency information to determine the relevance likelyhood of a particular template.
Such metrics require an existing set of validated templates, which may be gathered
by filtering a non discriminated set of templates by their measures importances.
