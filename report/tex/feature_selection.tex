\section{Template Extraction}
The approach we have taken to identify bird species from recordings is in
essence an image recognition task.
For this we use standard template matching techniques to match sections of
bird song in spectrograms against full song spectrograms.
This section details the procedures developed to clean up the spectrogram images
for section identification and extraction, and how these are filtered so as to
extract only those which are indicative of the relative species.

\section{Spectrogram Preprocessing}\label{sec:preproc}

Before any features may be extracted, their boundaries must be found.
This is a non-trivial problem due to undesireable noise and recording artefacts
present in each spectrogram.
The problem is considerably worsened by the inconsistency of these from sample
to sample.

The figure below shows an example of a noisy spectrogram.
Notice how the background noise makes it difficult to find the exact boundary of
some of the vocalisations.

show sgram with noise

Even with the quality prefiltering done when downloading recordings from
Xeno-canto, such noise levels remain common.

Because a high number of samples is required for our approach, we developed an
automatic noise reduction stage.
Standard computer vision techinques for noise reduction were used, as well as
techniques for discrete object identification.
A filter is first applied to the spectrogram, which reduces the noise in the
image by smoothing just enough until granular noise is reduced sufficiently (figure x).
The image is then thresholded using an adaptive thresholding algorithm (figure x).
We used adaptive thresholding due to the pixel intensity inconsistencies
throughout some of the spectrogram images.
Further noise reduction is then performed using erosion and dilation which
removes small segments and joins pixel groups which are in close proximity to
each other (figure x).

--\\

global parameters for sweeping noise reductions are suboptimal, what may work
for one recording might not work for another.

yet this is what we do, works ok, specific values were found through trial and
error.

objective is to process spectrograms to find contiguous blobs with similar
scope, that is, either phrases or individual vocalizations.

Some noise reduction steps are semi-automatic, such as adaptive thresholding.

a better approach would be to operate on a per-spectrogram basis, to find the
optimal parameters for each.

This is a non trivial problem.

conceptually the noise reduction algorithm would perform some parameter search
with a heuristic based on the dimensions and quantity of contiguous blobs, with
the aim of reducing the number of small blobs which may resemble noise or disjoint
parts of a single vocalization or segment.

The target scope should be specifiable, so that either individual vocalizations
or complete segments or songs could be extracted.

In some cases it might not be possible to achieve total correct segmentation.

Since different birds have different lengths for specific sounds or parts of song,
the spectral dimensions can not be generalized.
This means that each species will have different aims for quantity and dimensionality,
which must be constructed either by manual input or some feedback mechanism.

a feedback system can then be used to determine which type of segmentation works
best by filtering to various sizes and measuring the accuracy obtained after
classification.


\subsection{Basic Selection and Extraction}\label{sec:template_select}
Some effort is taken to reduce the number of templates extracted in this stage
to reduce the memory and therefore computation cost of template matching, as well
as to reduce the number of irrelevant or incorrectly formed features.
Incorrectly formed templates can consist of noise, partial and incomplete
sections of valid song partitions which may form due to imperfect preprocessing,
generic sections which do not strongly correlate to any one species,
and unique events which do not appear in any other recording, possibly due to
other species or external noises.

\textbf{is RF good at ignoring noise? I think so.?}

\textbf{speculative section on what makes a good template}

\subsubsection{Selection mechanism}
Complete elimination of all undesireable features is impossible without also
removing desireable features.
A balance must therefore be struck, in both preprocessing and selection stages.
Performing selection manually is extremely time consuming given the quantity
of pixel sections.
Selection is therefore done automatically by considering the dimensionality and
pixel values of each template.\\


Templates with the following properties are automatically rejected:
\begin{itemize}[noitemsep]
  \item \textbf{Area smaller than 50 pixels:} Templates with small dimensions
    are too small to be of any significance and are most likely noise or
    artefacts from preprocessing;

  \item \textbf{Area larger than 10000 pixels:} Templates with extremely large
    dimensions are most likely artefacts from preprocessing, high intensity
    noise or continuous external noises.
    It is possible for sections of song to be merged together into a single
    pixel region due to preprocessing errors.

  \item \textbf{Minimum pixel intensity greater than the average for the
    spectrogram:} Not sure why we do this
\end{itemize}


\subsubsection{Extraction mechanism}\label{sec:extract}
Once selection is performed for a particular pixel region, it's bounding box is
expanded by 10 pixels in each direction, and that region is extracted from the
original, unprocessed spectrogram.
This forms a single template.
The template is then blurred using Gaussian filtering with a sigma of 1.5.

The reason for blurring is to reduce the dimensionality of the template, which
brings performance boosts to template matching (described in
Section~\ref{sec:ccm}) since the template may now be reduced in size.

Blurring the template also improves it's generality, which makes matches more
likely.
Special consideration must be taken in order to ensure templates are not overly
generalised, as this will cause excess ambiguity amongst templates.



\subsection{Guided Template Elimination}
It is desireable to reduce the template count further by recognizing aspects which
make for good templates. This subsection outlines some speculative options for
selecting better templates, and reducing those which would have a low importance
score after training and classification.

\subsubsection{Image contrast}
It can be argued that templates with low local contrast contain insufficient
information to be meaningful in any way during template matching.
Templates with a low contrast match against much of any image, resulting in an
increase in noise.
Implicitly \textbf{is implicitly the right word?} such templates have a high
correlation with not only the species from which it was extracted but with all
species.
\textbf{show some images, maybe graphs of correlation}

\subsubsection{Spatial inclusion}
Due to the imperfect nature of the preprocessing methods used, gaps and
inconsistencies in structure appear in the thresholded spectrogram.
These inconsistencies are present also in repeated components in a bird song,
at all levels of granularity.
This causes multiple templates to be extracted for a single component in some
instances, and single larger blocks to be extracted in others.

In many of these cases, one template's bounding box intersects or is contained
entirely within another template's bounding box.
Merging these templates by extracting the bounding box of the union of the two
or more templates may result in more consistent extractions.
\textbf{show some images}

Similarly, templates which are sufficiently close to eachother may be merged,
but care must be taken not to form extremely large templates.

\subsubsection{Variation in granularity}
There exists a variance in granularity for the extracted templates, in which
some sections of song are mostly connected to form a single template, and others
are disconnected, leaving templates with single syllables \textbf{right word?}
and templates with entire sections of song.
This is a similar to the observation in \textbf{spatial inclusion}.

\textbf{what to do about it}
\textbf{is it a problem}


\subsubsection{inter-template correlation}
some templates may correlate with eachother, should they be merged?

It can also be argued that templates with little to know intercorrelation may be
independent anomalies such as noise in the signal or other sounds irrelevant to
the subject species.

\subsubsection{Species-specific template statistics}
It may be possible to use information regarding average dimensionality and mean
frequency information to determine the relevance likelyhood of a particular template.
Such metrics require an existing set of validated templates, which may be gathered
by filtering a non discriminated set of templates by their measures importances.
