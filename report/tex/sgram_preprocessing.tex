\subsection{Spectrogram Preprocessing}\label{sec:preproc}

Before any features can be extracted, their pixel boundaries must be found.
This is a non-trivial problem due to noise and recording artefacts
that may be present in each spectrogram.
The complexity of the problem is risen by the inconsistency of these from sample
to sample.

Figure~\ref{fig:sgram_noise} shows an example of a noisy spectrogram.
Notice how background noise makes it difficult to find the exact boundary of
some of the vocalisations.
\textbf{define noise, both background and unwanted signal also rewrite the notice}

\begin{figure}[h]
  \centering

  \caption{Considerably noisy spectrogram of a segment of xbird song.}
  \label{fig:sgram_noise}
\end{figure}

Even after quality prefiltering done when downloading recordings from
Xeno-canto, noise levels remain at undesireable levels.

Removing noise and unwanted artefacts manually is precise, although tedious.
Because a high number of samples is required for our approach, we developed an
automatic noise reduction stage, as doing so manually would be far too
impractical at this scale.
Standard computer vision techinques for noise reduction were used, as well as
techniques for discrete object identification.

\subsubsection{The mechanism}
A filter is first applied to the spectrogram, which reduces the noise in the
image by smoothing just enough until granular noise is reduced sufficiently
(Figure~\ref{fig:preproc_vis_filter}).
This is accomplished using a Gaussian filter with a 5x5 kernel and sigma of 0.
\textbf{sigma 0??}
Median filtering was also tested. \textbf{so what happened?}

The image is then thresholded using a thresholding algorithm
(Figure~\ref{fig:preproc_vis_thresh}).
Otsu's binarization thresholding technique was used to select an optimal
threshold value.
Because of pixel intensity inconsistencies throughout individual spectrograms,
specifically in areas where noise is prominent, an adaptive thresholding
algorithm was also tested.
This did not appear to provide generally better results.

Further noise reduction is then performed using dilation and erosion
(Figure~\ref{fig:preproc_vis_dilation1}),
which removes small segments and joins pixel groups which are in close proximity
to each other.
Notice that the order of operations is flipped here since we are working with
an inverted spectrogram.
Dilation and erosion is then performed a second time for larger segments
(Figure~\ref{fig:preproc_vis_dilation2}),
A kernel size of 3x3 and 7x7 were used for small and large segments, respectively.

Most remaining holes are then filled using a closing morphology algorithm,
using an ellipse of size 3x3 (Figure~\ref{fig:preproc_vis_closing}).

\begin{figure}[h]
  \centering
  \begin{subfigure}[b]{0.55\textwidth}

    \caption{Gaussian filter, $k=5x5, \sigma=0$}
    \label{fig:preproc_vis_filter}
  \end{subfigure}
  \begin{subfigure}[b]{0.55\textwidth}

    \caption{Otsu's threshold}
    \label{fig:preproc_vis_thresh}
  \end{subfigure}
  \begin{subfigure}[b]{0.55\textwidth}

    \caption{Dilation followed by erosion, $k=3x3$}
    \label{fig:preproc_vis_dilation1}
  \end{subfigure}
  \begin{subfigure}[b]{0.55\textwidth}

    \caption{Dilation followed by erosion, $k=7x7$}
    \label{fig:preproc_vis_dilation2}
  \end{subfigure}
  \begin{subfigure}[b]{0.55\textwidth}

    \caption{Closing, 3x3 ellipse}
    \label{fig:preproc_vis_closing}
  \end{subfigure}
  \caption{Preprocessing stages in order.}
  \label{fig:preproc_vis}
\end{figure}

Following these procedures the spectrogram is transformed into a binary image
consisting of discrete pixel segments.
This representation simplifies the extraction procedure as described in
Section~\ref{sec:extract}.


\subsubsection{Granularity considerations}\label{sec:granularity}
It is important to consider and evaluate the granularity to aim for when
isolating sections of song.
It is not clear if large sections of song would perform better than smaller,
individual vocalisations.
An evaluation should therefore be performed to determine the best granularity.

Trialing different granularities bears the weight of template matching for all
new templates, which is extremely time consuming.

Further, achieving consistent granularity across all spectrograms is not a trivial
task, and is certainly not possible if a single parameter set is used for
all spectrograms.

A loose aim is therefore taken to extract the smallest possible non-singular
vocalisations.
This of course does not always work, but the developed mechanism gives good
results.

\subsubsection{Quality and consistency}
It is difficult to arrive at a single set of optimal parameters that work
well across all spectrograms.
An adaptive method is therefore suggested (but not implemented):
Preprocessing parameters may be specified on a per-recording basis if prior
knowledge is gathered for expected vocalisation/section counts, template
dimensions, and frequency range.
Since there is no known structured source of data for this, it is necessary to
manually specify these on a per-species basis.
Alternatively a fully automatic method is conceivable by feeding back the
classification results with each granularity setting.
This however would be incredibly time consuming on standard hardware, and may
be sensitive to other factors in the classification pipeline.

Consistent quality becomes less of a concern as the number of samples used for
training increases.
It can be shown that as the sample count increases, the number of valid templates
tends to increase.
The number of noise also increases, however these do not intercorrelate as valid
templates do, and are handled well by less sensitive classifiers such as
random forests (\textcite{marko2004}).

Quality is therefore moderated after preprocessing in the selection stage as
described in Section~\ref{sec:template_select}.
