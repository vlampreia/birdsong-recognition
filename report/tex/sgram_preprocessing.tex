\section{Spectrogram Preprocessing}\label{sec:preproc}

Before any features may be extracted, their boundaries must be found.
This is a non-trivial problem due to undesireable noise and recording artefacts
present in each spectrogram.
The problem is considerably worsened by the inconsistency of these from sample
to sample.

The figure below shows an example of a noisy spectrogram.
Notice how the background noise makes it difficult to find the exact boundary of
some of the vocalisations.

show sgram with noise

Even with the quality prefiltering done when downloading recordings from
Xeno-canto, such noise levels remain common.

Because a high number of samples is required for our approach, we developed an
automatic noise reduction stage.
Standard computer vision techinques for noise reduction were used, as well as
techniques for discrete object identification.
A filter is first applied to the spectrogram, which reduces the noise in the
image by smoothing just enough until granular noise is reduced sufficiently (figure x).
The image is then thresholded using an adaptive thresholding algorithm (figure x).
We used adaptive thresholding due to the pixel intensity inconsistencies
throughout some of the spectrogram images.
Further noise reduction is then performed using erosion and dilation which
removes small segments and joins pixel groups which are in close proximity to
each other (figure x).

--\\

global parameters for sweeping noise reductions are suboptimal, what may work
for one recording might not work for another.

yet this is what we do, works ok, specific values were found through trial and
error.

objective is to process spectrograms to find contiguous blobs with similar
scope, that is, either phrases or individual vocalizations.

Some noise reduction steps are semi-automatic, such as adaptive thresholding.

a better approach would be to operate on a per-spectrogram basis, to find the
optimal parameters for each.

This is a non trivial problem.

conceptually the noise reduction algorithm would perform some parameter search
with a heuristic based on the dimensions and quantity of contiguous blobs, with
the aim of reducing the number of small blobs which may resemble noise or disjoint
parts of a single vocalization or segment.

The target scope should be specifiable, so that either individual vocalizations
or complete segments or songs could be extracted.

In some cases it might not be possible to achieve total correct segmentation.

Since different birds have different lengths for specific sounds or parts of song,
the spectral dimensions can not be generalized.
This means that each species will have different aims for quantity and dimensionality,
which must be constructed either by manual input or some feedback mechanism.

a feedback system can then be used to determine which type of segmentation works
best by filtering to various sizes and measuring the accuracy obtained after
classification.
