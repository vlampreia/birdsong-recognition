\section{Spectrogram Preprocessing}

show sgram with noise

Even with prefiltering performed when extracting recordings from xenocanto,
noisy samples are still common.
Because a high number of samples is required for our approach, we developed an
automatic noise reduction stage.

global parameters for sweeping noise reductions are suboptimal, what may work
for one recording might not work for another.

yet this is what we do, works ok, specific values were found through trial and
error.

objective is to process spectrograms to find contiguous blobs with similar
scope, that is, either phrases or individual vocalizations.

Some noise reduction steps are semi-automatic, such as adaptive thresholding.

a better approach would be to operate on a per-spectrogram basis, to find the
optimal parameters for each.

This is a non trivial problem.

conceptually the noise reduction algorithm would perform some parameter search
with a heuristic based on the dimensions and quantity of contiguous blobs, with
the aim of reducing the number of small blobs which may resemble noise or disjoint
parts of a single vocalization or segment.

The target scope should be specifiable, so that either individual vocalizations
or complete segments or songs could be extracted.

In some cases it might not be possible to achieve total correct segmentation.

Since different birds have different lengths for specific sounds or parts of song,
the spectral dimensions can not be generalized.
This means that each species will have different aims for quantity and dimensionality,
which must be constructed either by manual input or some feedback mechanism.

a feedback system can then be used to determine which type of segmentation works
best by filtering to various sizes and measuring the accuracy obtained after
classification.
