\section{Tuning}\label{sec:tuning}
why tune\\

when overfit\\


Tuning is generally performed by cross-validated grid search.
This involves running the same validation procedures described in
Section~\ref{sec:acc_eval}, but with a set of varying parameters.

Here we select the most significant parameters, as well as a set of discrete
values to test.
Grid search essentially involves testing all possible permutations of the
parameter value set.

This section discusses the results obtained by performing a grid search on
the following parameter set:
\begin{itemize}
  \item lala: [1, 2, 3]
\end{itemize}

Learning Curve

max features: maximum number of features per tree. auto select like sqrt takes 
total number of features sqrt.
higher numbers generally increase performance, but decrease diversity
features are chosen at random
small number of features reduces variance but increases bias of individual trees
ratio of noise/good features effects choice of max features
smaller maxfeatures means less chance of selecting good feature for split

num estimators: max number of trees to build. higher == better == slower
increasing this will not affect bias. test error is low due to variance reduction.
more trees = less variance
reduction in correlation of trees = less variance, achieved by random feature
selection

min sample leaf: size of leaf nodes, smaller = more sensitive to noise
dunno
affects tree depth?


cv used during search

max depth = none + minsamplssplit = 1 results in fully developed trees




\subsection{Gridsearch}
explain gridsearch and how to use it.
explain how we use it if we do.
show parameter ranges

validation curve
tune based on validation score
will be biased towards validation dataset
for better generalization estimation compute score on different set

we can use this to check over/under fitting though
check influence of single parameter on training and validation scores
if both are low, clf is undefitting
if training is high, valid is low, clf is overfitting
low training, high validation not possible usually

Analuse the results of parameter optimization
also measure overfitting (how)
show performance chages over changes in parameters

coarse-to-fine search gridsearch

plot it
            n estimators=500,
            max features='sqrt',
            min samples split=3,
            max depth
            bootstrap?

\subsection{Learning Rate}
learning curve
shows validation/training score for varying train sample count.
shows how clf may benefit from additional samples if at all
shows if clf suffers from variance or bias error
if validation and training score converge to a low value with more training samples,
we don't benefit. might have to change parameters or choose different clf which
has lower bias

if training score is higher than valid score, we benefit from adding training
samples probably, increasing generalisation


